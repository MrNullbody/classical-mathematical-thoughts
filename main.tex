\documentclass{main}
\usepackage{subfiles}
\title{\Huge\bfseries 经典王志超技巧}
\author{Anthropos Anonumos}
\date{}
\setCJKmainfont[Path=./fonts/,ItalicFont=simkai,BoldFont=simhei]{simsun}
\geometry{a4paper,scale=0.7,marginparwidth=3cm,headheight=10mm}
\pgfplotsset{compat=1.18}
\theoremstyle{definition}
\newtheorem{theorem}{定理}[section]
\newtheorem{proposition}[theorem]{命题}
\newtheorem{lemma}[theorem]{引理}
\newcommand{\dd}{\mathrm{d}}
\titleformat{\chapter}{\Huge\bfseries}{\huge\bfseries 第\thechapter 章}{0pt}{\vskip 20pt}
\renewcommand{\chaptermark}[1]{ \markboth{#1}{} }
\newcommand{\highlight}[2]{\colorbox{#1!17}{$#2$}}
\usetikzlibrary{tikzmark,calc,intersections}
\newcommand{\prefacename}{前言}
\newenvironment{preface}{
    \vspace*{\stretch{2}}
    {\noindent \bfseries \Huge \prefacename}
    \begin{center}
        \phantomsection \addcontentsline{toc}{chapter}{\prefacename}
        \thispagestyle{plain}
    \end{center}

}
{\vspace*{\stretch{5}}}
\begin{document}
\fancyhf{}
\pagenumbering{Roman}
\maketitle
\begin{preface}
    混沌初开,太始生数。昔伏羲仰观奎星圆曲之势,俯察龟甲纵横之纹,始作九九之术,此乃数术发轫之枢机。今有复旦八骏,振衣千仞之岗,濯足万里之流,聚首论道,欲以青春之笔,续写黄钟大吕之章。

观其论学,宛若伯牙抚琴,嵇康锻铁。欧几里得公理,如七弦泠泠,奏《阳春》《白雪》之曲;牛顿流数之术,似铁砧星火,铸天地运行之轨。昔张平子观天制仪,能测候风地动;今诸生执筹推演,欲解黎曼猜想。祖氏缀术,精算至圆率千分;笛氏坐标,妙合天圆地方之说。此非独术数之精微,实乃古今智慧之弦歌互答。

尝闻《九章》分畴,刘徽注海;《周髀》测影,赵爽释天。今诸子论学,每至月斜廊庑,犹闻争鸣之声:或辩康托尔无穷之奥,或析庞加莱猜想之玄。犹记程门立雪,杨时求教之诚;更慕濂溪观莲,茂叔格物之明。诸生切磋,有若弈秋对局,落子皆含宇宙;恰似庖丁解牛,奏刀必中音律。

昔者《易》云:“形而上者谓之道,形而下者谓之器。”数学之道,上通碧落,下贯黄泉。诸生今日所论,非止算筹之戏,实为探赜索隐之功。愿效张骞凿空,开辟数理新域;更期僧繇点睛,唤醒蛰龙腾渊。观此讨论班之设,不啻在杏坛弦诵,兰亭修禊之间也。

时逢元夜,又历春秋。黄浦江头,春潮暗涌;光华楼畔,桃李初芳。诸生勖哉!他日若遂凌云志,莫忘今朝灯下,曾有少年击节歌《九章》、抚掌论《几何》之时也。

是为序。
\begin{flushright}
编者

\today 于复旦园
\end{flushright}
\end{preface}
\tableofcontents
\newpage
\pagenumbering{arabic}
\setcounter{page}{1}
\pagestyle{fancy}
\fancyhead[RO,LE]{\thepage}
\fancyhead[LO]{\ifnum\value{chapter}>0 第\thechapter 章\fi\quad\leftmark}
\fancyhead[RE]{\rightmark}
\subfile{ch01}
\subfile{ch02}
\subfile{ch03}
\subfile{ch04}
\subfile{ch05}
\subfile{ch06}
\bibliographystyle{unsrt}
\bibliography{citations}
\end{document}
