\documentclass{main}
\usepackage{subfiles}
\title{\fontsize{40}{0}\bfseries 经典王志超技巧}
\author{\Large\scshape Anthropos Anonumos}
\date{}
\setCJKmainfont[Path=./fonts/,ItalicFont=simkai,BoldFont=simhei]{simsun}
\geometry{a4paper,scale=0.7,marginparwidth=3cm,headheight=10mm}
\pgfplotsset{compat=1.18}
\theoremstyle{definition}
\newtheorem{theorem}{定理}[section]
\newtheorem{proposition}[theorem]{命题}
\newtheorem{lemma}[theorem]{引理}
\newcommand{\dd}{\mathrm{d}}
\titleformat{\chapter}{\Huge\bfseries}{\huge\bfseries 第\thechapter 章}{0pt}{\vskip 20pt}
\renewcommand{\chaptermark}[1]{ \markboth{#1}{} }
\newcommand{\highlight}[2]{\colorbox{#1!17}{$#2$}}
\usetikzlibrary{tikzmark,calc,intersections}
\newenvironment{preface}[1][前言]{
    \newpage
    \vspace*{\stretch{2}}
    {\noindent \bfseries \Huge #1}
    \begin{center}
        \phantomsection \addcontentsline{toc}{chapter}{#1}
        \thispagestyle{plain}
    \end{center}

}
{\vspace*{\stretch{5}}}
\begin{document}
\fancyhf{}
\pagenumbering{Roman}
\maketitle
\begin{preface}[數學思想讚]
混沌初開,太始生數。昔伏羲仰觀奎星圓曲之勢,俯察龜甲縱橫之紋,始作九九之術,此乃數術發軔之樞機。今有復旦八駿,振衣千仞之崗,濯足萬里之流,聚首論道,欲以青春之筆,續寫黃鐘大呂之章。

觀其論學,宛若伯牙撫琴,嵇康鍛鐵。歐幾里得公理,如七絃泠泠,奏《陽春》《白雪》之曲;牛頓流數之術,似鐵砧星火,鑄天地運行之軌。昔張平子觀天制儀,能測候風地動;今諸生執籌推演,欲解黎曼度量。祖氏綴術,精算至圓率千分;笛氏坐標,妙合天圓地方之說。此非獨術數之精微,實乃古今智慧之弦歌互答。

嘗聞《九章》分疇,劉徽注海;《周髀》測影,趙爽釋天。今諸子論學,每至月斜廊廡,猶聞爭鳴之聲:或辯高斯無窮之奧,或析歐拉定理之玄。猶記程門立雪,楊時求教之誠;更慕濂溪觀蓮,茂叔格物之明。諸生切磋,有若弈秋對局,落子皆含宇宙;恰似庖丁解牛,奏刀必中音律。

昔者《易》云:「形而上者謂之道,形而下者謂之器。」數學之道,上通碧落,下貫黃泉。諸生今日所論,非止算籌之戲,實為探賾索隱之功。願效張騫鑿空,開闢數理新域;更期僧繇點睛,喚醒蟄龍騰淵。觀此討論班之設,不啻在杏壇弦誦,蘭亭修禊之間也。

時逢元夜,又歷春秋。黃浦江頭,春潮暗湧;光華樓畔,桃李初芳。諸生勖哉!他日若遂凌雲志,莫忘今朝燈下,曾有少年擊節歌《九章》、撫掌論《幾何》之時也。

是為序。
\begin{flushright}
    哈裤饭与深度求索

    2025年2月某日
\end{flushright}
\end{preface}
\begin{preface}[前言一]
几个月后,面对40多页的经典王志超技巧讲义,我会想起那个答应朋友参与经典数学思想讨论班的午后。

谈到这篇笔记的起源,就不得不提到将名字镌刻在这本笔记扉页上的男人——王志超。王志超老师是2024-2025秋季学期经典数学思想课程前6节课的老师。在他精确、快速地讲解下,我们成为了第一届成功把64课时的讲义在36课时内学完的学生。然,我等学生资质驽钝,在短短的课堂时间中无法彻底领会王志超先生的深意,又不愿在课下劳烦先生解惑,占用工作时间。于是自发地举行讨论班,在再次的讲解中领悟先生的深意。这篇笔记是讨论班内容的整理,七个人整理六个篇章编订而成,因而风格不同实属正常。

有人会问我为何要整理笔记,毕竟我等水平有限,种种表述也远不如书上精确、简洁。但对我来说,原因有二:

其一,经典数学思想讲义和老师课上的讲述省去了思考路径,使得其难以理解,我们在讨论班中讨论、发掘这些思考的路径和重点,于是想着将其记录下来,向后来者讲述我们的理解与思考。

其二,数学系学生在数学之海中遨游之时,遇到美妙、有趣的观点,总会将其记录下来与他人分享,也以此标定自己的学习轨迹。这篇笔记是我们自己学习这些知识时的思考,也是我们学习的标定。

正如上面所言,这篇笔记是我们学生在学习过程中所作,限于水平可能有种种错漏,读者不必将之当做严肃的老师,只需当做一个乐意分享的朋友即可。

这是我们第一次写笔记,若是多年以后回看或许会觉得幼稚,但对我们来说,这是学习数学过程中珍贵的体验。也愿每一位阅读这篇笔记的人都能在数学学习过程中找到自己的乐趣。
\begin{flushright}
    温雪

2025年2月12日
\end{flushright}
\end{preface}
\begin{preface}[前言二]
    数学的学习一直很顺遂,直到我们遇到了经典数学思想。

    我看数学书的时候会问自己,这个数学为什么是这样的。但是在\cite{cmt}这本讲义里,我发现,找到答案很困难。我们熟识的对象,讲义里会用动人的语言解释它们产生的动机,而到了我们未曾见过的概念,解释它们的语言就和它们本身一样玄妙。我时常呆坐在桌前,竭尽所能用我贫乏的想象力理解老师们脑海中奇幻的几何世界,补全讲义中缺失的逻辑拼图。

    毫不意外地,我的朋友们都遇到了这样的窘境。好在我们人多势众,总有人能解答其他人的疑惑。我们在课上小声地讨论,在课间三两成群地讨论,解决了眼前阻碍着我们理解的问题。可是我们始终没能解答自己心里最大的问题,这些数学为什么是这样的?为什么双重外积展开后没有外积?为什么分类二次曲面需要这么多不变量?为什么测地线要去掉切向曲率?解答这些问题需要更系统的讨论——在距离期末考试还有寥寥数周的时候,我们开始了第一次讨论班。

    每一个人专门向其他人讲解一章的内容,并把重点放在动机和思路上。这样,一两个小时恰好足够厘清一个章节。用一句话来评价我们的活动,那就是:顿悟的感觉真好。至少,我们的理解支撑我们有尊严地完成了期末试卷。

    可是讲义里还是有很多问题没有解决,有些证明的步骤没有过去,有些概念带着很深的背景。正值寒假,我们觉得不如重新回顾一遍自己负责的内容,将那些因为时间原因留有遗憾的细节补全,用文档的形式互相分享,顺便精进\TeX 技术。成果斐然。为了更好地保存这一段独特经历的成果,我将我们所写的文档汇编成了这份近五十页的文件。

    在这份文件里,我们尽可能地展现了我们心目中数学应该有的样子。如你所见,与讲义上的形式差异巨大。这可能意味着两点:
    \begin{enumerate}
        \item 我们的数学品味和理解尚不成熟,只能用稚嫩的语言描述我们对内容的理解;
        \item 我们因为已有知识的差异,与编写讲义的老师们站在不同的视角看待同样的数学。
    \end{enumerate}
    但无论怎样,你能看到这份文件,是因为我们希望能通过分享自己的理解,为那些认为自己还没有完全理解的同学(包括我们自己),提供一些思考、回顾的参考。这份文档在各位眼中想必是粗鄙不堪,还请大家多多指导。

    参与此次活动的同学共有八名,其中奶龙因为某些原因没有参与文档的编写,但也是讨论活动的重要参与者。编写人员:哈裤饭——第一章,ATRENTYA——第二章,ideal——第三章,温雪——第四章,15531及拉师傅——第五章,\'ecoute——第六章。文件采用\LaTeXe 技术排版,源代码见\href{https://github.com/MrNullbody/classical-mathematical-thoughts}{GitHub}。
    \begin{flushright}
        \'ecoute

        2025年2月12日
    \end{flushright}
\end{preface}
\tableofcontents
\newpage
\pagenumbering{arabic}
\setcounter{page}{1}
\pagestyle{fancy}
\fancyhead[RO,LE]{\thepage}
\fancyhead[LO]{\ifnum\value{chapter}>0 第\thechapter 章\fi\quad\leftmark}
\fancyhead[RE]{\rightmark}
\subfile{ch01}
\subfile{ch02}
\subfile{ch03}
\subfile{ch04}
\subfile{ch05}
\subfile{ch06}
\bibliographystyle{unsrt}
\bibliography{citations}
\end{document}
