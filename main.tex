\documentclass{main}
\usepackage{subfiles}
\title{\fontsize{40}{0}\bfseries 经典王志超技巧}
\author{\Large\scshape Anthropos Anonumos}
\date{}
\setCJKmainfont[Path=./fonts/,ItalicFont=simkai,BoldFont=simhei]{simsun}
\geometry{a4paper,scale=0.7,marginparwidth=3cm,headheight=10mm}
\pgfplotsset{compat=1.18}
\theoremstyle{definition}
\newtheorem{theorem}{定理}[section]
\newtheorem{proposition}[theorem]{命题}
\newtheorem{lemma}[theorem]{引理}
\newcommand{\dd}{\mathrm{d}}
\titleformat{\chapter}{\Huge\bfseries}{\huge\bfseries 第\thechapter 章}{0pt}{\vskip 20pt}
\renewcommand{\chaptermark}[1]{ \markboth{#1}{} }
\newcommand{\highlight}[2]{\colorbox{#1!17}{$#2$}}
\usetikzlibrary{tikzmark,calc,intersections}
\newenvironment{preface}[1][前言]{
    \newpage
    \vspace*{\stretch{2}}
    {\noindent \bfseries \Huge #1}
    \begin{center}
        \phantomsection \addcontentsline{toc}{chapter}{#1}
        \thispagestyle{plain}
    \end{center}

}
{\vspace*{\stretch{5}}}
\begin{document}
\fancyhf{}
\pagenumbering{Roman}
\maketitle
\begin{preface}[數學思想讚]
混沌初開,太始生數。昔伏羲仰觀奎星圓曲之勢,俯察龜甲縱橫之紋,始作九九之術,此乃數術發軔之樞機。今有復旦八駿,振衣千仞之崗,濯足萬里之流,聚首論道,欲以青春之筆,續寫黃鐘大呂之章。

觀其論學,宛若伯牙撫琴,嵇康鍛鐵。歐幾里得公理,如七絃泠泠,奏《陽春》《白雪》之曲;牛頓流數之術,似鐵砧星火,鑄天地運行之軌。昔張平子觀天制儀,能測候風地動;今諸生執籌推演,欲解黎曼度量。祖氏綴術,精算至圓率千分;笛氏坐標,妙合天圓地方之說。此非獨術數之精微,實乃古今智慧之弦歌互答。

嘗聞《九章》分疇,劉徽注海;《周髀》測影,趙爽釋天。今諸子論學,每至月斜廊廡,猶聞爭鳴之聲:或辯高斯無窮之奧,或析歐拉定理之玄。猶記程門立雪,楊時求教之誠;更慕濂溪觀蓮,茂叔格物之明。諸生切磋,有若弈秋對局,落子皆含宇宙;恰似庖丁解牛,奏刀必中音律。

昔者《易》云:「形而上者謂之道,形而下者謂之器。」數學之道,上通碧落,下貫黃泉。諸生今日所論,非止算籌之戲,實為探賾索隱之功。願效張騫鑿空,開闢數理新域;更期僧繇點睛,喚醒蟄龍騰淵。觀此討論班之設,不啻在杏壇弦誦,蘭亭修禊之間也。

時逢元夜,又歷春秋。黃浦江頭,春潮暗湧;光華樓畔,桃李初芳。諸生勖哉!他日若遂凌雲志,莫忘今朝燈下,曾有少年擊節歌《九章》、撫掌論《幾何》之時也。

是為序。
\end{preface}
\begin{preface}
\begin{flushright}
编者

\today 于复旦园
\end{flushright}
\end{preface}
\tableofcontents
\newpage
\pagenumbering{arabic}
\setcounter{page}{1}
\pagestyle{fancy}
\fancyhead[RO,LE]{\thepage}
\fancyhead[LO]{\ifnum\value{chapter}>0 第\thechapter 章\fi\quad\leftmark}
\fancyhead[RE]{\rightmark}
\subfile{ch01}
\subfile{ch02}
\subfile{ch03}
\subfile{ch04}
\subfile{ch05}
\subfile{ch06}
\bibliographystyle{unsrt}
\bibliography{citations}
\end{document}
