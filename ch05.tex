\documentclass[./main.tex]{subfiles}
\begin{document}
\chapter{射影几何}
进入第五章射影几何, 我们将会拥有射影坐标系这一强大的工具, 运用解析的方法, 处理与点线结合有关, 较为复杂的平面几何问题. 

讲义在本章容易带来一些理解上的困难, 在此给一些难点作进一步阐释. 
\section{射影对象}
\subsection{射影直线}
    讲义中提供了四种认识射影直线$\mathbb{R}P ^{1}$的观点, 关键的想法是建立二维平面中过原点的直线与某个等价类的一一对应. 这依赖于代表元的选取. 以下两点是两种不同的代表元选取方式.
    \begin{itemize}
        \item 与单位圆上的点建立对应是自然的. 需要注意的是, 我们考虑的对象是直线, 所以每组对径点对应同一条直线. 但这样的操作并没有带来实际意义上的简化, 也很难推广到三维的情况 (我们马上会看到这一点) .
        \item 一个好的做法是``尽量''与$y=1$上的点建立对应,这样我们解决了大部分直线的对应, 剩下$y=0$这条直线对应到$\left[1,0\right]$. 为了一致性, 我们将$\left[1,0\right] $视作$\left[\infty,1 \right]$.  (这在记号上当然是不被允许的, 但初次学习时这样能带来理解上的简化, 所有代表元即为$\left[t,1\right] $, t取值范围为全体实数与无穷.) 这样一来, $\mathbb{R}P ^{1}$的结构就变得相当清楚了.
    \end{itemize}
$\mathbb{R}P ^{1}$相较于我们主要的研究对象, 射影平面$\mathbb{R}P ^{2}$而言, 是一个简单的铺垫. 当然, 二者的思想与技巧是类似的, 只是后者具体操作上要麻烦一些. 
\subsection{射影平面}
与$\mathbb{R}P ^{1}$类似地, 在$\mathbb{R}^{3}$中, 我们同样渴望建立过原点的直线与某个等价类的一一对应. 与上面两种方式相对应产生了以下两种直观的对应. 
\begin{itemize}
    \item 与单位球上的点建立对应. 同样的, 我们需要把对径点``粘合''起来. 遗憾的是, 在$\mathbb{R} ^{3}$中, 这并不可能做到.
    \item ``尽量''与平面$z=1$上的点建立起对应, 剩余操作的细节处理见讲义, 此处不再赘述. 与直觉相符的是, $\mathbb{R} P^{2}$由一个平面与其无穷远直线构成. 
\end{itemize}
\subsection{\texorpdfstring{\(\mathbb{R}P ^{2}\)}{RP\^{}2}上的射影直线}
本小节的关键是记号的明确. 值得注意的是, 射影直线与$\mathbb{R}^{3}$中的过原点平面一一对应, 用平面的法向量来表示射影直线. 如此一来, 点的叉乘得到直线, 直线的叉乘得到点 (在三维空间中前者是两个直线的方向向量叉乘得到过二者平面的法向量, 后者则是两个平面法向量叉乘得到交线的方向向量.) 初次学习时不妨先返回三维空间思考, 有时能够带来帮助. Desargues定理是一个简单的应用案例. 其告诉我们, 在处理点共线, 线共点一类问题中, 射影几何相较于平面几何相当方便, 因为``射影直线必共线''减少了许多讨论情况的功夫. 
\section{射影坐标系与射影坐标}
在具体的操作与验证上, 本节并没有实际上的困难, 这里再补充两点批注.
\begin{itemize}
\item 关于为什么要选取额外的射影点$[u]$,其用途仅仅是固定$[a],[b],[c]$的代表元. 需要注意的是, $[a]$与$[2a]$表示的是同一个点, 在未确定代表元的情况下, 射影坐标会变得无法确定. 

\item (5.7)中的$\left(\cdot,\cdot,\cdot\right) $均表示混合积. (5.7)即行列式乘法 (若遗忘建议复习) 
\end{itemize}
\section{平面射影几何的内容}
Pappus定理与先前的Desargues定理一样, 是一个和点与直线结合关系有关的结论, 此处略过.

第四调和点的证明 (命题17) 告诉我们: 选择一个好的射影坐标系相当重要. 令$[x]$, $[y]$, $[u]$, $[v]$这条直线第一个分量为零, 对之后在这条直线上建立射影坐标系的简化是显著的. 

最后来看$\mathbb{R}P ^{2}$的对偶原理. 根据前面点线结合的论证, 这是相当自然的 (点共线和线共点是对偶的) . 由此我们可以从已知结论中得到一些对偶定理. 
\end{document}
