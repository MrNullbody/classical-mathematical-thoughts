\documentclass[./main.tex]{subfiles} 
\begin{document}
\chapter{等距变换与仿射变换}
\section{等距变换}
数学家只要等距变换就可以了, 而普通人想的可就多了, 比如平移、反射和旋转. 
\subsection{我们为什么研究等距变换}
等距变换的定义在这里不再赘述, 但研究等距变换的动机却是值得好好考虑的问题. 在笔者看来, 最大的动机来自于对``全等''的刻画. 我们在初中就学过全等, 两个通过平移、旋转、翻折可以重合的图形被称为全等的, 但为什么是这些操作?这些操作有什么共同点?这一类变换有什么性质?这是我们开始研究等距变换的原因. 
\subsection{我们怎么研究等距变换}
\subsubsection{平面等距变换}
我们从平面等距变换开始我们的研究. 讲义上给出了等距变换的定义后一步一步地分析、挖掘其性质, 最终指出其代数表示. 但在笔者看来, 既然研究等距变换的最初动机是研究其在几何上的性质, 那自然应该从其几何性质着手. 事实上, 不难看出平移和旋转都是两个反射的复合, 于是自然猜测所有等距变换都可以写成若干个反射的复合, 进而给出如下定理:
\begin{theorem}[三反射定理]
    平面上的任一等距变换都可以表示为小于等于三个反射的复合. 
\end{theorem}
证明的大致思路是在三个不共线的点确定一个平面的基础上, 证明每次反射可以增加一个不共线的不动点. 证明细节可参见\cite{gos}.

于是我们自然的有等距变换是线性变换, 进而与书上相同的推导可以得到其表达式为
\[
X'=A\cdot X+B,
\]
其中$A$为2×2正交矩阵, $B,X,X'$为2维列向量. \marginpar{\footnotesize 这里$A$的行列式为1代表其为旋转或平移, -1代表其为翻折或滑反射. }
\subsubsection{空间等距变换}
自然地, 在研究完平面等距变换后, 我们想将其推广至空间中, 研究空间等距变换. 我们想考虑三反射定理在三维空间中的推广, 即三维空间中任何等距变换也可以被写成若干个 (空间中) 反射的复合. 但在开始讨论之前, 理应先明确空间中的反射和旋转是什么. 一般来说, 空间中的反射指镜面反射, 即关于平面作反射, 旋转则是以一条射线为轴作逆时针旋转. 则可以容易地将三反射定理推广至空间中. 
\begin{theorem}
    空间中任意等距变换都可以被写成小于等于四个反射的复合. 
\end{theorem}
于是我们同样有
\[
X'=A\cdot X+B,
\]
其中$A$为3×3正交矩阵, $B,X,X'$为3维列向量. 

但与平面不同, 前面的定理还不能让我们看清三维空间中的等距变换有哪些. 注意到$B$其实是平凡的, 不妨通过平移使得$B=0$, 即考虑有至少一个不动点的情况. \marginpar{\footnotesize 这种变换叫做正交变换.}
\begin{lemma}
    在上述条件下, 有一条直线$l$在等距变换$\Phi$下保持不变. 
\end{lemma}
引理证明由3×3矩阵必有特征向量且正交矩阵特征值非零得到. 
\begin{theorem}
    三维空间中的正交变换只有恒等映射, 镜面反射, 旋转反射, 旋转四种
\end{theorem}
\begin{proof}
    证明通过代数的观点来看是简单的. 考虑引理1中给出的直线$l$及其正交补, 则有两空间正交, 只需分别考虑两空间上的变换即可. 

    二维空间中正交变换为恒等映射, 反射和旋转, 直线上则为恒等映射和反射. 于是两两复合依次得到三维空间中的恒等映射, 镜面反射, 旋转和旋转反射. 
\end{proof}
\subsection{关于群}
在上述的推导过后, 群的事实就很显然了, 在这里单开一节只是为了凸显群的尊贵地位 (bushi) 
\section{仿射变换}
\subsection{我们为什么研究仿射变换?}
与等距变换不同, 仿射变换的原始定义让人看起来一头雾水——什么叫做``把直线变为直线的可逆变换''. 在笔者看来, 仿射变换研究的是较等距变换更为普遍的情形, 即直线型之间的变换. 将直线变为直线保证了直线型在变换后仍为直线型, 可逆则保证了其是平面 (或空间) 上的一个一一对应, 这样的变换正是我们要研究的不同方向上的伸缩变换和拉伸变换. 
\subsection{我们怎么研究仿射变换?}
\subsubsection{仿射变换的代数化}
与等距变换不同, 仿射变换的本质没有那么容易在几何上刻画, 因而不得不将其代数化后研究. 课本上给出了一个技术化的推导, 证明了在仿射坐标系$\{O;\epsilon_1,\epsilon_2\}$下, 仿射变换的坐标变换公式是
\[
X'=A\cdot X +B 
\]
其中$X',X,B$为二维列向量, 且$detA\not=0$同时满足上述公式的变换必然是仿射变换. 
\subsubsection{两种分解方式}
在书中给出了两种分解方式, 可以帮助我们更进一步地看清保持一点不变的仿射变换\marginpar{\footnotesize 实质上通过平移可使得一点不变, 这与我们在等距变换章节中研究正交变换是同样的.}的本质. 
\begin{theorem}[讲义定理15]
    平面上保持一点不变的仿射变换可以分解为水平拉伸变换、垂直拉伸变换与伸缩变换的复合. 
\end{theorem}
定理的证明相当技术化, 在此不再赘述, 但有意思的是这个定理说明了什么. 就像书上图4.10所表现的, 水平拉伸变换决定的是竖直方向的坐标倾斜的方向, 垂直拉伸变换则是水平方向坐标的倾斜, 伸缩变换则表现为长度的变化. 因此我们很自然地看出了仿射变换的实质就是改变两个坐标轴 (或者说仿射坐标系) 的方向和度量. 

而如下的定理给出了另一个刻画. 
\begin{theorem}[讲义4.4定理16/习题1]
    保持一点不变的仿射变换$\phi$可分解为$\phi=S_2\circ S_1$其中$S_1$为正交变换, $S_2$为沿两个相互正交方向的伸缩变换. 
\end{theorem}
证明不再赘述. 但这个定理实质上说明的是仿射变换其实就是等距变换伸缩后的结果, 将等距变换和仿射变换联系在了一起. 
\subsubsection{仿射不变量}
更多的时候我们将仿射变换作为工具来处理难以解决的问题. 所以我们需要研究什么东西在仿射变换下是不变的, 进而通过仿射变换计算. 事实上, 根据上面我们对于仿射变换本质的探究, 我们自然可以感觉到仿射变换下的线段比 (即分比) 和面积比应当是不变的 (因为正如我们上面所说, 仿射变换的本质是等距变换和伸缩的复合) . 具体代数化、严格化的证明在书4.3.4一节中提到. 
\section{结语}
本篇章中笔者省略了相当多的证明细节, 因为在笔者看来对于等距和仿射变换这两种几何变换, 理解、感受其本质远比死抠细节重要. 如果因此造成阅读障碍, 使得无法流畅地阅读笔记, 笔者在此表示抱歉. 

读者若觉得自己与笔者意见相左, 也不必大肆批判或是怀疑自己. 毕竟笔者水平有限, 在一些观点的陈述上难免有疏漏. 读者只需将其当做学习过程中供参考、交流的材料即可. 
\end{document}
